\chapter{Literature Review}
\label{ch:chapter2}

\section{Introduction}

This chapter reviews existing research and methodologies related to threat modeling, penetration testing, and predictive security approaches. The literature is organized into three key areas: (1) traditional threat modeling frameworks, (2) AI and automation-based methodologies, and (3) integration of threat modeling with penetration testing.

\section{Foundational Threat Modeling Approaches}

Xu et al. introduced an automated security test generation approach based on formal threat models using Predicate/Transition (PrT) nets. The framework achieved about 95\% executable attack paths and a 90\% mutant kill rate on Magento and FileZilla Server, demonstrating high test effectiveness \cite{xu2013}.

Marback et al. proposed integrating Data Flow Diagrams (DFDs) with STRIDE methodology to identify threats and transform them into executable test cases. The approach improved test coverage and detected multi-step attacks on real web applications \cite{marback2013}.

Palanivel and Selvadurai presented a risk-driven testing approach integrating Extended Finite State Machines (EFSMs) with STRIDE-based threat modeling. The methodology reduced test cases by around 20\% without compromising coverage \cite{palanivel2014}.

Rahim et al. applied STRIDE-DREAD-based threat modeling to Water Grid Systems (WGS) using the Microsoft Threat Modeling Tool. They identified 154 potential threats with highest risks in Tampering (score 14), DoS (score 13), and Repudiation (score 12) \cite{rahim2022}.

Alozie examined threat modeling frameworks (PASTA, STRIDE, Attack Trees) for healthcare cybersecurity, recommending STRIDE-based models with automation and ML support for improved healthcare system resilience \cite{alozie2024}.

\section{Automated and AI-Enhanced Threat Modeling}

Granata and Rak conducted a systematic analysis of open-source automated threat modeling tools. Microsoft's tool demonstrated the highest automation (88 threats), OWASP Dragon identified 31 threats, and PyTM detected 91 detailed threats using CAPEC and MITRE data \cite{granata2020}.

Shin et al. proposed the Actionable Intelligence-Oriented Cyber Threat Modeling Framework, implementing the TIME prototype to automate correlation of assets, vulnerabilities, and external CTI data using NIST SCAP standards \cite{shin2022}.

Dekker and Alevizos developed TIBSA (Threat-Intelligence Based Security Assessment), a six-step methodology integrating CTI and causal graph modeling to assess cyber risks while minimizing uncertainty and bias \cite{dekker2023}.

Smith et al. proposed Predictive Behavioral Mapping (PBM) framework for ransomware detection. PBM achieved 98.6\% detection accuracy and 1.8\% false positive rate, significantly outperforming conventional approaches \cite{smith2023}.

Bin Sarhan and Altwaijry presented a machine learning-based insider threat detection framework using Deep Feature Synthesis (DFS), PCA, and SMOTE on the CERT r4.2 dataset. The SVM classification model achieved 100\% accuracy \cite{binsarhan2023}.

\section{Integration of Threat Modeling with Penetration Testing}

Alharbi et al. evaluated the security of a Siemens SICAM CMIC Remote Terminal Unit (RTU) by integrating formal threat modeling with practical black-box penetration testing using Nmap, OWASP ZAP, Nikto, Nexpose, and SQLMap \cite{alharbi2020}.

Chu discussed automation of penetration testing using Belief-Desire-Intention (BDI) architecture with ontology-based reasoning. The automated approach reduced execution time from 179 seconds to 52 seconds on Metasploitable2 \cite{chu2021}.

Huang and Zhu proposed PenHeal, a two-stage LLM framework for automated penetration testing. PenHeal improved vulnerability detection coverage by 31\%, remediation effectiveness by 32\%, and reduced remediation costs by 46\% \cite{huang2024}.

Chauhan analyzed penetration testing's impact on mitigating insider threats through Systematic Literature Review (SLR) methodology across IEEE Xplore, ScienceDirect, SpringerLink, ACM Digital Library, and Google Scholar \cite{chauhan2024}.

Sanagana integrated threat modeling with Next-Generation Firewall (NGFW) architectures, proposing a proactive framework combining DPI, IPS, and Application Awareness with threat intelligence feeds \cite{sanagana2023}.

Maniraj et al. showcased OWASP ZAP for web application security testing. Experimental results showed 88-94\% coverage, identifying 120-150 vulnerabilities including 20-35 critical ones in 25-35 minutes \cite{maniraj2024}.

\section{Research Gap}

The reviewed literature demonstrates significant advancements in integrating threat modeling with automated security testing, penetration testing, and AI-driven defense mechanisms. While approaches such as PrT-net-based automation, STRIDE-DFD threat modeling, and risk-driven EFSM testing improved accuracy and efficiency, they often required high modeling expertise and faced scalability challenges. AI-enhanced and CTI-integrated frameworks like TIBSA, TIME, and PenHeal achieved automation and predictive intelligence but remained limited by data quality, computational overhead, and domain specificity. Tools like OWASP ZAP and NGFW integrations showed strong practical applicability but suffered from false positives and limited coverage of complex attack vectors. These findings reveal the need for a unified, intelligent framework that bridges automated threat modeling, AI-based prediction, and dynamic penetration testing.

\section{Available Solutions Analysis}

\subsection{Commercial Solutions}

\textbf{Microsoft Threat Modeling Tool:} Automates threat identification using DFDs and STRIDE framework with intuitive graphical interface but operates primarily as standalone design-time tool with limited integration for continuous monitoring.

\textbf{Cigent Platform:} Combines automated threat discovery with risk quantification, integrating with existing security tools but lacking native penetration testing capabilities.

\textbf{ThreatModeler:} Automates threat modeling across SDLC supporting STRIDE, PASTA, and OCTAVE methodologies with CI/CD integration but limited penetration testing beyond basic vulnerability scanning.

\textbf{IriusRisk:} Provides automated threat modeling with DevSecOps focus, extensive threat pattern library, and API-based integrations but no native penetration test execution.

\subsection{Open-Source Solutions}

\textbf{OWASP Threat Dragon:} Free tool supporting STRIDE-based threat identification through visual DFD modeling but lacking automated testing capabilities.

\textbf{PyTM:} Code-based framework allowing programmatic architecture definition with CAPEC and MITRE ATT\&CK databases but remaining primarily modeling tool.

\textbf{Threagile:} Risk-centric tool using YAML-based architecture definitions with quantitative risk scoring but no active testing capabilities.

\textbf{OWASP ZAP:} Extensively used for web application security testing with automated vulnerability scanning but lacking threat modeling capabilities.

\textbf{Metasploit Framework:} Comprehensive penetration testing framework with vast exploit collection but no threat modeling or risk prioritization mechanisms.

\subsection{Gap Analysis}

Critical gaps identified include: Integration Gap (disconnect between threat identification and validation), Automation Gap (lack of end-to-end continuous validation), Intelligence Gap (limited AI/ML for predictive analysis), Continuous Monitoring Gap (point-in-time assessments), Unified Risk Context Gap (fragmented results), and Feedback Loop Gap (limited capability for results to update models).

\section{Tools Background}

\subsection{Threat Modeling Frameworks}

\textbf{STRIDE:} Microsoft's mnemonic-based classification system covering Spoofing, Tampering, Repudiation, Information disclosure, Denial of service, and Elevation of privilege.

\textbf{DREAD:} Risk assessment model scoring threats on Damage potential, Reproducibility, Exploitability, Affected users, and Discoverability.

\textbf{PASTA:} Seven-stage risk-centric methodology integrating business objectives with technical security analysis.

\textbf{MITRE ATT\&CK:} Globally-accessible knowledge base of adversary behaviors organizing attack techniques into tactical categories.

\textbf{CAPEC:} Comprehensive dictionary of known attack patterns maintained by MITRE.

\subsection{Penetration Testing Tools}

\textbf{Metasploit:} World's most widely used framework with over 2,000 exploits and 500 payloads.

\textbf{Nmap:} Industry-standard tool for network discovery supporting various scanning techniques.

\textbf{OWASP ZAP:} Integrated tool acting as man-in-the-middle proxy for web application testing.

\textbf{Burp Suite:} Comprehensive platform combining manual and automated testing capabilities.

\textbf{Nikto:} Open-source web server scanner checking over 6,700 potentially dangerous files.

\textbf{SQLMap:} Automated SQL injection detection tool supporting wide range of database systems.

\textbf{Hydra:} Fast login cracker supporting numerous protocols for authentication testing.

\subsection{Vulnerability Assessment Tools}

\textbf{Nessus:} Widely deployed scanner with extensive database (over 165,000 plugins) covering CVEs, configuration issues, and compliance violations.

\textbf{OpenVAS:} Open-source framework with continuously updated feed of Network Vulnerability Tests.

\textbf{Nexpose:} Enterprise-grade solution combining comprehensive scanning with risk-based prioritization.

\subsection{Integration Frameworks}

\textbf{PTES:} Comprehensive framework defining standard methodology for penetration testing engagements with seven phases.

\textbf{OWASP Testing Guide:} Resource for web application security testing maintained by OWASP community.

\textbf{Jenkins:} Open-source automation server for CI/CD pipelines enabling automated security testing.

\subsection{AI and ML Tools}

\textbf{TensorFlow and PyTorch:} Leading frameworks for developing ML models for anomaly detection and threat prediction.

\textbf{Scikit-learn:} Python library for data mining and machine learning including classification algorithms.

\textbf{NLP Tools:} Tools like spaCy and transformers for automated analysis of security documentation.

\subsection{Threat Intelligence Platforms}

\textbf{MISP:} Open-source platform for sharing, storing, and correlating Indicators of Compromise.

\textbf{STIX/TAXII:} OASIS standards for representing and exchanging cyber threat intelligence.

\textbf{AlienVault OTX:} Community-driven platform providing access to millions of threat indicators.

\subsection{Virtualization Technologies}

\textbf{Docker:} Containerization platform for creating consistent, reproducible testing environments.

\textbf{Kubernetes:} Container orchestration platform for managing large-scale testing operations.

\textbf{VirtualBox and VMware:} Virtualization platforms for isolated testing environments.

\section{Summary}

The tools and frameworks reviewed represent current state-of-the-art in threat modeling, penetration testing, and security automation. Each tool addresses specific aspects yet typically operates independently. Understanding their capabilities and limitations provides foundation for designing integrated framework leveraging their strengths while addressing identified gaps.